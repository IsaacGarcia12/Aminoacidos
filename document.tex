\documentclass{article}
% Comment the following line to NOT allow the usage of umlauts
\usepackage[utf8]{inputenc}
% Uncomment the following line to allow the usage of graphics (.png, .jpg)
%\usepackage{graphicx}

% Start the document
\begin{document}

% Create a new 1st level heading
\section{Introducción}
a anotación de secuencias de aminoácidos tiene como función clasificar y explicar el funcionamiento de la proteína; cuyos factores principales, se encuentran las estructuras primaria y secundaria, su representación en el espacio 3D, taxonomía, dominio, familia, entre otros; el conjunto de todos estos factores sirve para investigaciones más complejas, como lo son el desarrollo de vacunas o de medicinas. Es por eso que este trabajo se considera que podría ser un buen auxiliar para las investigaciones médicobiológicas.\\\\
El problema principal que se propone resolver, es el de documentar, de manera general, las propiedades de una secuencia de aminoácidos nueva. Además de lo antes mencionado y, tomando en cuenta que actualmente existen diferentes sistemas que permiten conocer una parte básica de las proteínas gracias al alineamiento, ninguno de ellos se encarga de proponer una anotación completa de manera automática, de ahí el motivo por el que surge el interés de realizar el presente trabajo.
\section{Justificación} 
Año tras año se descubren aproximadamente 7,107,096 proteínas nuevas [5] de las cuales se desconoce, en un inicio, la estructura primaria, secundaria y componentes que la conforman.\\\\
El proceso de documentación para estas nuevas proteínas se realiza, hasta el día de hoy, de manera manual o con herramientas que no brindan mucho apoyo más allá de estructuras básicas como hélices u hojas B; por lo que proponemos un sistema que otorgue una descripción más detallada de las proteínas, con el fin de ayudar a los investigadores, logrando agilizar el proceso de clasificación de las proteínas.\\\\
Actualmente se poseen herramientas que sirven para alinear la secuencia de dos proteínas para descubrir que tanto se parecen o realizan la documentación de éstas de manera manual. [1, 4]\\\\
Cabe mencionar que la originalidad de este trabajo consiste en el desarrollo e implementación de un sistema de anotaciones automático que sea de utilidad en el campo de la investigación médico-biológica.\\\\
Se pretende elaborar este trabajo terminal con la finalidad de innovar en el campo de la investigación de documentación de proteínas aportando un nuevo instrumento auxiliar que pueda agilizar los procesos de investigación. Esto facilitará la sistematización de toda la información que se puede obtener a partir de una secuencia de aminoácidos.\\\\
Los usuarios que se verán directamente beneficiados con la elaboración de este proyecto, son aquellos involucrados en el estudio e investigación del área de Médico-Biológica, mediante el uso de técnicas de la Bioinformática.\\\\
En este proyecto se utilizarán herramientas como Pymol [6], VMD [7] que son empleadas para la Bioinformática y que apoyarán en el desarrollo y realización de las pruebas necesarias. En cuanto al desarrollo del sistema se emplearán conocimientos adquiridos en las áreas de Sistemas Complejos, Sistemas Web, Bioinformática y Ciencias de la Computación. E n cuanto a el lenguaje de programación que se usará será python. Por otro lado, la forma en que se planea trabajar en el proyecto es de acuerdo al tiempo marcado en los cronogramas establecidos en los anexos, tiempo que resultará adecuado para cumplir con las exigencias del proyecto, según la metodología a emplear.\\\\
Cabe destacar que esta propuesta propone una perspectiva diferente respecto a la presentación de la información que los usuarios la requieren.
\section{Estado del arte}
\section{Marco Teórico} 
% Uncomment the following two lines if you want to have a bibliography
%\bibliographystyle{alpha}
%\bibliography{document}

\end{document}
